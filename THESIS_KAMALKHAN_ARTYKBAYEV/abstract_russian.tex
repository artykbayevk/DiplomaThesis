\newpage
\pagestyle{plain}

{\selectlanguage{russian}
\begin{center}
    \Large
    \textbf{Аннотация}
\end{center}
Обнаружение и распознавание логотипов по прежнему представляет большой интерес для исследователей в области поиска информации, ведь за счёт реализации этого метода можно эффективно и точно идентифицировать источник определенного документа. Данный тезис вносит свой вклад в проектирования широкомасштабных систем, которые будут способны обнаружить позицию логотипа из любого документа или изображения, а далее провести этот логотип через классификатор, который в свою очередь с эффективной точностью определит класс этого логотипа. Классификатор был реализован через сверточные нейронные сети, который в данном контексте обучался на выборке, где были логотипы. Для этого классификатора была выбрана определенная выборка данных, чтобы пресечь всем проблемам в распознавании образов и объектов. Но основная цель работы заключается в том, чтобы в конечном итоге определить имеется ли какой-нибудь логотип на изображении, если он есть, то какого он класса.
}