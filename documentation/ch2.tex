\section{Introduction}\label{sec:2.1}
\vspace{-0.5cm}
\noindent In this chapter, we will explain the main problems that researchers faced when they recognize and detect logos. And also, briefly explain how you can solve difficulties of this complex process. We will also present you the content of the thesis, which will briefly clarify what will be shown in the following chapters.


%%%%%%%%%%%%%%%%%%%%%%%%%%%%%%%%%%%%%%%%%%%%%%%%%%%%%%%%%%%
\vspace{-0.3cm}
%%%%%%%%%%%%%%%%%%%%%%%%%%%%%%%%%%%%%%%%%%%%%%%%%%%%%%%%%%%


\section{Statement of the Problem}\label{sec:2.2}
\vspace{-0.5cm}
\noindent Soon

\vspace{-0.5cm}
\par
\noindent Soon

\vspace{-0.5cm}
\par
\noindent Sooon 


\vspace{-0.3cm}
%%%%%%%%%%%%%%%%%%%%%%%%%%%%%%%%%%%%%%%%%%%%%%%%%%%%%%%%%%%
\section{Thesis Organization}\label{sec:2.3}
\vspace{-0.5cm}
\noindent The structure of the thesis is organized as follows:

\vspace{-0.8cm}
\begin{itemize}
  \item  In Chapter 3, a general review of the most important adaptive filters used for echo cancellation application is presented. %The chapter begins with the basic proportionate-type adaptive filters followed by $l_1$-norm and $p$-norm based sparse adaptive filters.
\vspace{-0.3cm}
  \item In Chapter 4, the proposed algorithm is presented. A review of the VSSLMS algorithm and a broad concept of the $p$-norm constraint are provided. The mean square convergence analysis and a stability criterion of the proposed algorithm are also carried out and presented.
\vspace{-0.3cm}
  \item In Chapter 5, an experimental study is provided in order to compare the performance of the proposed filter with other $l_1$-norm and $p$-norm based sparse adaptive filters  in the context of AEC.
\vspace{-0.3cm}
  \item In Chapter 6, conclusions and a discussion on possibilities for future work are provided.
\end{itemize}

\vspace{-0.5cm}
