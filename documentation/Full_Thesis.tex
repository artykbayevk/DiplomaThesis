\documentclass[14pt,epsfig,times]{report}
\usepackage{EMU_Thesis_ikeV2}
\usepackage{graphicx,epsfig,times}
\usepackage{amsmath}
\usepackage{enumerate}
\usepackage{booktabs,slashbox}
\usepackage{color}
\usepackage{multirow}
\usepackage{dcolumn}
%%%%%%%%%%%%%%%%%%%%%
\usepackage[inner=0.75in,outer=0.65in,top=0.5in,bottom=0.75in]{geometry}

\title{Logo detection and recognition using CNN\vspace{-2cm}}

\author{Artykbayev Kamalkhan Serzhanovich}

\begin{document}
\pagenumbering{roman}
\makemstitle    % For M.S. theses
% \makephdtitle  % For Ph.D. theses
% \maketitle

% \begin{center}
% \newpage
% \vspace*{10cm} \center{\textbf{}}
% \end{center}

\begin{abstract}
\par
\begin{center}
\centering Artykbayev Kamalkhan Serzhanovich
\end{center}

\par
\begin{center}
    \centering B.A. Thesis, 2018
\end{center}

\par
\begin{center}
\centering Thesis supervisor: Senior Lecturer MSc. Konstantin Latuta
\end{center}

\par
\noindent \textbf{Keywords}: Logo detection, Logo recognition, Computer Vision, Machine Learning, Convolution Neural Network, Classification, Recurrent Neural Network, Pattern Recognition, Object Recognition,Data augmentation

\noindent This thesis describes the research work carried out to fulfill the Bachelor in Computer Science at the Suleyman Demirel University. Research was in Technopark at Suleyman Demirel University and was supervised by Konstantin Latuta. 
Logo detection and recognition continues to be of great interest to the document retrieval community as it enables effective identification of the source of a document. This paper contributes the design of the system able to detect the logo of any product from the documents and images after that recognize it from the archive via the convolutional neural network. For detecting and recognize of logos implemented via convolutional neural network, which creates initial classification to determine the presence of the logo on the document or image.As regards to the former, a collection of logos was designed and implemented to train the classier, to identify and to extract the logo features which were eventually used for logo detection and recognition. The latter regards the detection of logos from an input image. In particular, the experimental study aimed to detect if the input image contains one or more logos and to decide which logos are contained.

\end{abstract}

\begin{acknowledgements}
\vspace{-0.5cm}
\noindent I thank  the merciful and all-knowing, for sparing my life in sound health and giving me the opportunity to accomplish this thesis.
\vspace{-0.5cm}
\par
\noindent I wish to express my deepest gratitude to my supervisor Senior Lecturer MSc. Konstantin Latuta for his guidance, advice, criticism, encouragement and insight throughout the research.

\vspace{-0.5cm}
\par
\noindent I am highly indebted to my parents for their encouragement, support and unlimited love.


\vspace{-0.5cm}
\par
\noindent Finally, i wish to extend a special thanks to my colleagues for their valuable support and company. They really made my life a fabulous one.
\end{acknowledgements}

% \begin{center}
% \newpage
% \vspace*{9cm} \center{Dedicated to my parents}
% \end{center}



\tableofcontents
% \listoffigures
% \listoftables

\begin{symbols}
\sym{MSE}{Mean-square-error}
\sym{CNN}{Convolutional neural network}
\sym{RNN}{Recurrent Neural Network}
\sym{CV}{Computer Vision}
\sym{ML}{Machine Learning}
\sym{LRT}{Learning Rate}
\sym{Conv}{Convolutional layer}
\sym{Pool}{Pooling layer}
\sym{ReLU}{Rectified Linear Unit}
\sym{Softmax}{Normalized Exponential Function}
\sym{Sigm}{Special case of logistic function}

% \sym{$\delta$}{Regularization parameter} \sym{$\varepsilon$}{Reweighting parameter} \sym{$\gamma$}{Control parameter} \sym{$\kappa$}{Adjusting parameter} \sym{$\lambda$}{Eigenvalue} \sym{$\mu$}{Step-size} \sym{$\wp$}{Correlation parameter} \sym{$\sigma ^2$}{Variance} \sym{$\textbf{I}$}{Identity matrix} \sym{$\textbf{Q}$}{control matrix} \sym{$N$}{Filter length}  \sym{$\textbf{R}$}{Autocorrelation matrix} \sym{$\textbf{w}$}{Tap weights vector} \sym{$\textbf{x}$}{Tap input vector} \sym{AEC}{Acoustic echo canceller} \sym{AWGN}{Additive white Gaussian noise} \sym{ACGN}{Additive correlated Gaussian noise}\sym{CS}{Compressive sensing} \sym{DSP}{Digital signal processing}  \sym{FIR}{Finite impulse response} \sym{IPNLMS}{Improved proportionate normalized least-mean-square} \sym{LMS}{Least-mean-square} \sym{MSE}{Mean-square-error}  \sym{MSD}{Mean square deviation} \sym{NLMS}{Normalized least-mean-square} \sym{NNCLMS}{Non-uniform norm constraint LMS} \sym{PNLMS}{Proportionate normalized least-mean-square} \sym{RZA-LMS}{Reweighted zero-attracting LMS}  \sym{SNR}{Signal-to-noise ratio} \sym{VSSLMS}{Variable step size LMS} \sym{ZA-LMS}{Zero-attracting LMS}
\end{symbols}

\chapter{INTRODUCTION}\label{ch1}
\pagenumbering{arabic}
\section{Overview} \label{sec:1.1}
\vspace{-0.5cm}

\par Object recognition and object detection are ones of the lasting and most important goals in the computer vision.Because of this problem in a extremely considerable range of usage. For example, in copyright detection, contextual advertise placement, vehicle logo for an intelligent AI-based traffic-control system and brand detecting in social media. As well as these algorithms have many applications in location recognition, advertisement, and marketing. Presently advertising is a very powerful tool for income and attracting of customers.For this reason, the analysis of the brand and mention on different resources are very important and primary tasks for business analysts.In order to captive, attractive their customers and make better decisions, companies needs for analyzing the presence of their logos in photos, videos and another type of contents. Logos help to evaluation of identity between something.[1604.06083] [1701.02620] [1711.09822]

\vspace{-0.5cm}
\par The logo mainly includes text and graphical symbols. In such cases, when the logo is in different parts of the image, the logo is inverted, the logo is distorted, as well as changed in size - recognition and definition of the logo is a very problematic and difficult task. For example, logos on their clothes, which are often deformed, which complicates its detection and recognition. [1511.02462]

\vspace{-0.5cm}
\par Recent breakthroughs in deep learning improve recognition models with very extremely minimal loss function.Models which was created for recognition based on neural networks have excellent accuracy, speed, as well as these models, have the ability to be really smart. In short, the ultimate goal of the system, which is based on the recognition model, is to create a method that defines logos accurately and continuously learn from new logos.[1711.09822]

\vspace{-0.3cm}
%%%%%%%%%%%%%%%%%%%%%%%%%%%%%%%%%%%%%%%%%%%%%%%%%%%%%%%%%%%%%%%%%%%%%%
\section{Related Works}\label{sec:1.2}
\vspace{-0.5cm}
\noindent In literature, I can find many works on logo detection, logo extraction, logo classification , logo retrieval and logo recognition.Research that has been done on for the last 20 years and that are related to logos have been done with datasets, which consist too small data. For example, "Flickr-Logos" dataset, which includes only 32 logos, which distributed to 5644 objects on 8240 images.Obviously available public datasets not enough for creating real-time detector machine. And this machine, will not be able to fully use its potential of detection with a neural network, due to the lack of data about other logos.[1511.02462]

%%%%%%%%%%%%%%%%%%%%%%%%%%%%%%%%%%%%%%%%%%%%%%%%%%%%%%%%%%%%%%%%%%%%%%
\vspace{-0.3cm}
%%%%%%%%%%%%%%%%%%%%%%%%%%%%%%%%%%%%%%%%%%%%%%%%%%%%%%%%%%%%%%%%%%%%%%
\par Most of recent applications of detection and recognition of object have been based mostly on Scale-Invariant Image features.Scale-invariant image features algorithm provide us transformations and representations to gradients of images. These gradients are invariant to affine type transformations and despite the conditions.Models that were made based on SIFT, basically make a better that part of the picture that is specifically different from the rest of the content.At the moment, although there is a plural number of methods for logo recognition, the performance, and power of Convolutional Neural Network is growing very strongly in the field of computer vision. After all, CNN solves many of the problems of the basic classical computer vision algorithms and includes a large range of uses in the recognition of images and objects. The structure of convolutional neural networks is very hierarchical and multilayered, as well as it is designed so that the pattern can be recognized only from the pixels themselves.[1604.06083]


%%%%%%%%%%%%%%%%%%%%%%%%%%%%%%%%%%%%%%%%%%%%%%%%%%%%%%%%%%%%%%%%%%%%%%
\vspace{-0.3cm}
%%%%%%%%%%%%%%%%%%%%%%%%%%%%%%%%%%%%%%%%%%%%%%%%%%%%%%%%%%%%%%%%%%%%%%
\par In this [bianco2017deep] paper they propose a method of logo detection and recognition with using main deep learning algorithms.Their recognition and detection process have given with pipeline, which consists main 5 step. These steps: taking an image, doing object proposal, cropping to regions, passing through the trained convolutional neural network and making a prediction.This algorithm recognizes logos well even if they are not exactly localized in the image.The neural network was trained and tested on the "FlickrLogos-32" database.For improving result they trained CNN with very differently benefits. As an example, for avoiding overfitting they used class - balancing in every batch. Also, they confirm sample-weighting and add a new class  'no logo', which includes only images without any logos.



%%%%%%%%%%%%%%%%%%%%%%%%%%%%%%%%%%%%%%%%%%%%%%%%%%%%%%%%%%%%%%%%%%%%%%
\vspace{-0.3cm}
%%%%%%%%%%%%%%%%%%%%%%%%%%%%%%%%%%%%%%%%%%%%%%%%%%%%%%%%%%%%%%%%%%%%%%

\par Also, this [1511.02462] paper presents a method that works perfectly with logo recognition, and returns the bounding box of the found logo. In particular, recognition of the logo has broad application and uses it in many areas. To protect people intellectual property, logo recognition is the most convenient and effective tool.As mentioned earlier, in the area of logo recognition and identification, most tools have a very small dataset.But researchers from this article had presented a large-scale database, which has 160 classes distributed among 130608 objects. This dataset is really huge and it is called "LOGO-net 160". For cropping the image into regions and search regions of interest(RoI) they used selective search algorithm, that efficient for this type of tasks. After features extraction with CNN, fully connected two layers divided into softmax predictor and bounding box regressor. This mathematical operations provided us classification of logo and it's position on image.



%%%%%%%%%%%%%%%%%%%%%%%%%%%%%%%%%%%%%%%%%%%%%%%%%%%%%%%%%%%%%%%%%%%%%%
\vspace{-0.3cm}
%%%%%%%%%%%%%%%%%%%%%%%%%%%%%%%%%%%%%%%%%%%%%%%%%%%%%%%%%%%%%%%%%%%%%%

\par Guys from this [1604.06083] paper demonstrate method for recognition, which based on Region-based Convolutional Networks.A distinctive feature of this approach to solving the problem is the recognition of multiple objects in the image.

%%%%%%%%%%%%%%%%%%%%%%%%%%%%%%%%%%%%%%%%%%%%%%%%%%%%%%%%%%%%%%%%%%%%%%
\vspace{-0.5cm}
%%%%%%%%%%%%%%%%%%%%%%%%%%%%%%%%%%%%%%%%%%%%%%%%%%%%%%%%%%%%%%%%%%%%%%

\chapter{PROBLEM STATEMENT AND THESIS ORGANIZATION}\label{ch2}
\section{Introduction}\label{sec:2.1}
\vspace{-0.5cm}
\noindent Soon
\vspace{-0.3cm}
%%%%%%%%%%%%%%%%%%%%%%%%%%%%%%%%%%%%%%%%%%%%%%%%%%%%%%%%%%%
\section{Statement of the Problem}\label{sec:2.2}
\vspace{-0.5cm}
\noindent Soon

\vspace{-0.5cm}
\par
\noindent Soon

\vspace{-0.5cm}
\par
\noindent Sooon 


\vspace{-0.3cm}
%%%%%%%%%%%%%%%%%%%%%%%%%%%%%%%%%%%%%%%%%%%%%%%%%%%%%%%%%%%
\section{Our Contributions}\label{sec:2.3}
\vspace{-0.5cm}

\noindent Soon

\vspace{-0.3cm}
%%%%%%%%%%%%%%%%%%%%%%%%%%%%%%%%%%%%%%%%%%%%%%%%%%%%%%%%%%%
\section{Thesis Organization}\label{sec:2.4}
\vspace{-0.5cm}
\noindent The structure of the thesis is organized as follows:

\vspace{-0.8cm}
\begin{itemize}
  \item  In Chapter 3, a general review of the most important adaptive filters used for echo cancellation application is presented. %The chapter begins with the basic proportionate-type adaptive filters followed by $l_1$-norm and $p$-norm based sparse adaptive filters.
\vspace{-0.3cm}
  \item In Chapter 4, the proposed algorithm is presented. A review of the VSSLMS algorithm and a broad concept of the $p$-norm constraint are provided. The mean square convergence analysis and a stability criterion of the proposed algorithm are also carried out and presented.
\vspace{-0.3cm}
  \item In Chapter 5, an experimental study is provided in order to compare the performance of the proposed filter with other $l_1$-norm and $p$-norm based sparse adaptive filters  in the context of AEC.
\vspace{-0.3cm}
  \item In Chapter 6, conclusions and a discussion on possibilities for future work are provided.
\end{itemize}

\vspace{-0.5cm}


%\chapter{REVIEW OF DEEP LEARNING AND PATTERN RECOGNITION ALGORITHMS}\label{ch3}
%
\section{Introduction}\label{sec:3.1}
\vspace{-0.5cm}
\noindent This chapter provides a review of the well-known algorithms of image segmentation, pattern recognition, exhaustive search and deep learning methods.We will also explain how the main image segmentation methods work and how they developed in computer vision. Also will demonstrated methods forward and backpropagation. Between this two process, you can see the optimization process, which try to minimize function of error. 

%%%%%%%%%%%%%%%%%%%%%%%%%%%%%%%%%%%%%%%%%%%%%%%%%%%%%%%%%%%
\vspace{-0.3cm}
%%%%%%%%%%%%%%%%%%%%%%%%%%%%%%%%%%%%%%%%%%%%%%%%%%%%%%%%%%%




\section{Computer Vision and Pattern Recognition} \label{sec:3.2}
\vspace{-0.5cm}
\noindent For a person, the perception of the outside world with your own eyes is a very simple task, be looking at any 2- or 3-dimensional object, you can safely tell about its shape and external structure.Looking at the crowd of people, the human brain can easily calculate the number of objects, can tell about their shape and condition. But what about the computer? Will the computer be able to handle the processing of objects that people see? Will the computer be able to find the difference between very similar objects? In this matter will help discipline called computer vision.This area is very closely related to signal processing, image processing, and video recording. As well as it includes machine learning with pattern recognition.Along with other Sciences like text processing and audio processing, science tries to create the ideal artificial intelligence that can think and act like a human.Image processing not only includes the transformation of images into a more comfortable and desired type but also this area along with computer vision will be able to show what is inside the image. Image processing not only includes the transformation of images into a more comfortable and desired look but also this area along with computer vision will be able to show what is inside the image. Also, this area helps in capturing movements inside the picture.[CVPR]
Understanding what exactly is happening on images and perception of this process is an important process in AI. Draw conclusions depending on what you see is a fairly simple process for a person, but not for the computer. Since a computer without any reason cannot understand the essence of the process. The problem in object recognition is the appearance of these objects in new forms or compositions because the pre-built model cannot cope with it, because it has not seen the object in this format. These new formats can be represented as an object in the expanded state, or it can be simply in motion. A huge number of new forms and aspects makes the object recognition a practically impossible task.[introTOCV]
%\vspace{-0.5cm}
%\par 

%\vspace{-0.5cm}
%\noindent 
%%%%%%%%%%%%%%%%%%%%%%%%%%%%%%%%%%%%%%%%%%%%%%%%%%%%%%%%%%%%%%%%%%%%
\vspace{-0.3cm}
%%%%%%%%%%%%%%%%%%%%%%%%%%%%%%%%%%%%%%%%%%%%%%%%%%%%%%%%%%%%%%%%%%%%


\section{Selective Search}\label{sec:3.3}
\vspace{-0.5cm}
\noindent Exhaustive search helps to find parts of the image where you want to consider the potential parts of the desired object.Although this model works well with specially selected objects, it has a number of drawbacks that significantly affect the detection of logos.After all, the search for every possible object has the ability to be impossible. To solve this, we can use selective search.To improve the whole process and the data set for testing, we can use a combined method where we will use both methods described above. Since the number of possible objects will be more and less real and possible.Diversity in this task plays an important role, as we can cover more and more possible variants of this logo in the image.Since selective search is more useful to us, it will be helpful to familiarize yourself with its dependencies.The first and most important factor is to cover as many scales as possible because the logo can be small or large. We may not warn that. After all, the situation can be quite different. Also can make problems of objects which have no clear borders, for this reason, it is necessary to look through all options of the sizes of an object.Also, it should be noted that there is no exact and general solution of searches of any objects. It is impossible to make such a general detection system. Well, at the moment of course. For this reason, you should also look at the variety of objects and their contours, which can be very important in training. Speed is an important factor when searching for possible objects in an image. After all, such systems are built to determine the objects on the camera in a short period of time.This method is exclusive in that it is possible to configure this so that it worked by concentrating on the object and not on its borders.[ssForSegmentation]

\section{Image Segmentation Methods}\label{sec:3.4}
\vspace{-0.5cm}
\noindent In practice, the importance and value is not always fully the image itself, namely what are the specific parts of the image, and sometimes just the number of channels of the image.The first and one of the most important technologies for understanding what is happening inside this image is segmentation.Since only a segmentation can be divided into important and different parts.After all, it helps to understand the image inside the image, as well as to extract useful information for us.  These aspects are extremely important for programs where image recognition is paramount.For all these reasons, it can be understood that segmentation is a very important discipline within computer vision, and in turn, segmentation has a huge number of difficulties in implementing many methods.In short, segmentation is important for recognition, because it can pull out those areas that are very important for humans. And are the basis for all methods of recognition of contours and objects.There are many types of segmentation and a huge number of places where you can use them.One of the most common methods is threshold segmentation.The basis of this algorithm creates a segmentation of the image by its regions.This method searches for a threshold by a specific criterion to create a grayscale that will be distributed from other colors. This method sets a specific threshold for pixels and depending on the condition they change from 0 to 255 in grayscale.You can also mark a method called edge segmentation. This method is particularly the fact that he refers to the saturation of gray on the borders of any object.In the discipline of computer vision and related industries, there is no single segmentation method that can work in all cases. To use the segmentation method correctly, you need to consider the advantages and disadvantages. After all, each method will lead in different ways depending on the situation and the state in the image. And it is also very important to apply the correct parameters of segmentation methods. Since the parameters play a significant role in the algorithm.[1707.02051]

\vspace{-0.5cm}
\par
\noindent Segmentation, by itself, is splitting the image into several areas, depending on their structure, size, and saturation of any particular colors. These areas can include grouped pixels, which represent the object itself, and can represent a variety of shapes, such as an arc, circle, or just a line. Developed regions can be simple lines or full-fledged objects that can have boundaries separating them from other content. Since the area of interest may not cover the entire image, we are interested in using segmentation in such cases. Segmentation has two main goals that it pursues. The first is to expand the image to the desired regions. The second task is to change the representation.Considering the simplest cases, when the interesting part of the image is very different from the rest, the segmentation will not be a problem. After all, the area of our interest, especially its color and saturation help to clearly separate it from the rest of the image. After all, the rest of the area does not have similar components as in the desired image area.But there are also severe cases where the boundaries are strongly distorted and erased as the color saturation is very similar, and the components do not differ from each other.Considering the second objective pursued by the segmentation can be sure it will ultimately give us a richer and more precise representation of the object within the image. Here our task is to gather pixels into one whole, into a more integral area, which is much useful and important for future research, because we create a clearer outline of the object. The perspective of an image can serve both as a useful tool and a very strong drawback since the borders can be clearly highlighted or even erased in the image. \textbf{Here will be one two images, seg1.png and seg2.png}Typically, classic segmentation techniques may not work well for images where the boundaries between the desired features are blurred and blurred, making this work practically impossible because the pixels are too similar and the features cannot be separated or isolated.To divide the image into several parts according to the regions, have to be extremely homogeneous as the level of gray. After all, black-and-white images are easier to work with due to algorithms. As well as the color and texture of the image are also important when dividing by regions. Neighboring areas of the desired object should have very different characteristics and features because the uniformity prevents the algorithm. Also, borders of the object should be evenly distributed, and also they should not be torn or distorted. Achieving all the above characteristics gives a certain amount of difficulty, after all, how would the objects did not have their uniform or completely, they still have dire and slits, which interfere with segmentation algorithms, making a homogeneous region in a heterogeneous region. Also, our eye can also be mistaken in terms of the homogeneity of the object inside the image, because sometimes there may be holes or cuts that are not subject to our eye, so the number of pixels that we can not see, can interfere with the segmentation.[ch10]


%%%%%%%%%%%%%%%%%%%%%%%%%%%%%%%%%%%%%%%%%%%%%%%%%%%%%%%%%%%%%%%%%%%%
\vspace{-0.3cm}
%%%%%%%%%%%%%%%%%%%%%%%%%%%%%%%%%%%%%%%%%%%%%%%%%%%%%%%%%%%%%%%%%%%%

\subsection{Thresholding}\label{sec:3.4.1}
\vspace{-0.5cm}
\noindent In the methods of segmentation are the segmentation types of image in parallel. The most common and easiest method is to segment an image using a threshold. This method is based on the use of gray color. After all, we know that translating the image into a black and white contour with it is more convenient to work with than 3 color channels. This method segments the image based on image separation by saturation and grayscale. It is able to divide image according to its local threshold, which is automatic, depending on the distribution of the white and black color. And also, you can split the image using a global defect which can be defined as static and automatic and manually. It can also be noted that the threshold can be dynamic because it changes from area to area by an image.Global threshold divides the image into the desired area and its background, which he considered not similar to the area of interest to us. The local one does this by going through the image, and depending on the situation and position, select a threshold to be divided into the main part and the background part.THE most common and convenient method of threshold segmentation is - Otsu method. This method uses the interclass variance to separate areas of an image. The method is special and distinctive in that it selects only the global threshold. And the threshold is chosen by the maximum dispersion between all classes inside the image. The segmentation method has found extensive application due to the fact that it is very simple to calculate and does not require costly calculations and calculation when the algorithm itself. Also, due to a simple calculation and increases the speed of the algorithm. This algorithm can work very well when the boundaries between the object and the background are separated by an accurate and bold contrast line. In such cases, you can obtain accurate segmentation results for the image. But in the opposite case, when the boundaries are erased, this method is not able to cope, because it will not know exactly what is the object and its background. Noise can easily interfere with this method. Because the noise erases the boundaries. And uniformity also can ruin the quality of the method is the segmentation threshold.This method is effective in combined use with other methods.[1707.02051]

%%%%%%%%%%%%%%%%%%%%%%%%%%%%%%%%%%%%%%%%%%%%%%%%%%%%%%%%%%%%%%%%%%%%
\vspace{-0.3cm}
%%%%%%%%%%%%%%%%%%%%%%%%%%%%%%%%%%%%%%%%%%%%%%%%%%%%%%%%%%%%%%%%%%%%

\subsection{Clustering Methods}\label{sec:3.4.2}
\vspace{-0.5cm}
\noindent Clustering is a very powerful and unpredictable method in machine learning and computer vision. In computer vision, or rather in segmentation, it works by splitting image vectors into groups called clusters. Since clustering methods are not the same type, we can consider several types of clustering methods, but the essence of its work is based on similar points, which are very similar, and then they are grouped into separate clusters.The main problem of clustering is the correct splitting of the image into the correct sets of vectors. In this case, these vectors must be collected to have a similar value in the numbers, which means their structure must be similar. In these vectors consists mainly of the pixels of the image. Also, components can be indicators of the intensity in a given area, and 3 channel parameters that are related to each other. Texture, namely their calculated values can also be components. To associate pixels in groups, you can use any component or parameter that combines these pixels into a single value. Due to this, it is possible to find the associated objects and re-create the segmentation for the pixel count.[ch10]The\textit{ least squares error} is one of the most common measures to compare clusters that use the traditional method to break into groups. Clustering involves the process that determines the number of clusters $\kappa$. the same is created the number of groups from $C_{1}$ to $C_{\kappa}$. Each such cluster has its own personal measure of average $m_{\kappa}$. The formula of the error, said earlier, is calculated as follows:
\begin{equation}
D = {\sum\limits_{k=1}^{\kappa}} {\sum\limits_{x_i \in{C_{\kappa}}}^{\i}}{\Vert x_i - m_k \Vert}^{2}
\end{equation}

\vspace{-0.5cm}
\noindent This formula shows how close this object's data is to a particular cluster. This procedure will help you to see all the possible options for partitioning into K-th number of clusters. As a result, it will find the best option to minimize our error function D. the Disadvantage of this method is that it is impossible to calculate everything. For this reason, they find the closest number in value and divide the rest of the objects into clusters. It is also very difficult to find a global and optimal variant of the error function iteratively, for this reason, they usually resort to the random selection of clusters and selection of their mean values for further calculations. This method is called k-means clustering. There is also a method that is different from it. It is called isodata clustering. It uses a similar method of splitting and merging.This method is based on creating groups from their distance from the center of a particular cluster. Clusters are initialized randomly and iteratively go through the positions to find the most optimal point at which the error function will be reaching.[ch10]

\vspace{-0.5cm}
\noindent This method is the simplest and fastest, and most efficient for large datasets. Because it's easy to scale, it's very responsive for large datasets. The iterative nature of this method makes the optimization process easier and more convenient for calculations. But it also has a number of drawbacks, such as the number of clusters and the parameters by which these clusters need to be calculated. The iterative method is bad because every step goes through the whole sample, which is very expensive and time-consuming to calculate. There is also a problem with non-convex clusters, as they are difficult and impossible to calculate.[1707.02051]


%%%%%%%%%%%%%%%%%%%%%%%%%%%%%%%%%%%%%%%%%%%%%%%%%%%%%%%%%%%%%%%%%%%%
\vspace{-0.3cm}
%%%%%%%%%%%%%%%%%%%%%%%%%%%%%%%%%%%%%%%%%%%%%%%%%%%%%%%%%%%%%%%%%%%%

\subsection{Edge detection}\label{sec:3.4.3}
\vspace{-0.5cm}
\noindent Image edge performance greatly reduces the amount of data to be processed, but it retains the necessary information regarding the shapes of objects in place. This image explanation is easily incorporated into a large number of object recognition algorithms used in computer vision along with other image processing applications. The main property of edge detection method is its ability to remove fine edge line with a good orientation such as
well, as well as more literature on edge detection has been available in the last three decades. On the other hand, there is no General performance directory to evaluate the performance of boundary detection methods. The performance of edge detection methods is always evaluated in person and depends on its application.Edge detection is the primary tool for image segmentation. Methods of determining the boundaries to transform the source image at the image edges due to the change of the grayscale in the image. In image processing, especially in computer vision, edge detection considers the localization of important variations in gray level images and the detection of the physical and geometric properties of scene objects. This is the fundamental process of defining and outlining the object and the boundaries between objects and the background in the image. Edge detection
the most familiar approach to detect significant discontinuities in intensity values.

%%%%%%%%%%%%%%%%%%%%%%%%%%%%%%%%%%%%%%%%%%%%%%%%%%%%%%%%%%%%%%%%%%%%
\vspace{-0.3cm}
%%%%%%%%%%%%%%%%%%%%%%%%%%%%%%%%%%%%%%%%%%%%%%%%%%%%%%%%%%%%%%%%%%%%

\subsection{Histogram-based methods}\label{sec:3.4.4}
\vspace{-0.5cm}
\noindent  Soon

%%%%%%%%%%%%%%%%%%%%%%%%%%%%%%%%%%%%%%%%%%%%%%%%%%%%%%%%%%%%%%%%%%%%
\vspace{-0.3cm}
%%%%%%%%%%%%%%%%%%%%%%%%%%%%%%%%%%%%%%%%%%%%%%%%%%%%%%%%%%%%%%%%%%%%

\subsection{Compression-based methods}\label{sec:3.4.5}
\vspace{-0.5cm}
\noindent Soon


%%%%%%%%%%%%%%%%%%%%%%%%%%%%%%%%%%%%%%%%%%%%%%%%%%%%%%%%%%%%%%%%%%%%
\vspace{-0.3cm}
%%%%%%%%%%%%%%%%%%%%%%%%%%%%%%%%%%%%%%%%%%%%%%%%%%%%%%%%%%%%%%%%%%%%

\subsection{Dual clustering method}\label{sec:3.4.6}
\vspace{-0.5cm}
\noindent Soon

%%%%%%%%%%%%%%%%%%%%%%%%%%%%%%%%%%%%%%%%%%%%%%%%%%%%%%%%%%%%%%%%%%%%
\vspace{-0.3cm}
%%%%%%%%%%%%%%%%%%%%%%%%%%%%%%%%%%%%%%%%%%%%%%%%%%%%%%%%%%%%%%%%%%%%




\section{Supervised Learning}\label{sec:3.5}
\vspace{-0.5cm}
\noindent Soon

%%%%%%%%%%%%%%%%%%%%%%%%%%%%%%%%%%%%%%%%%%%%%%%%%%%%%%%%%%%%%%%%%%%%
\vspace{-0.3cm}
%%%%%%%%%%%%%%%%%%%%%%%%%%%%%%%%%%%%%%%%%%%%%%%%%%%%%%%%%%%%%%%%%%%%




\section{Optimization}\label{sec:3.6}
\vspace{-0.5cm}
\noindent Soon

%%%%%%%%%%%%%%%%%%%%%%%%%%%%%%%%%%%%%%%%%%%%%%%%%%%%%%%%%%%%%%%%%%%%
\vspace{-0.3cm}
%%%%%%%%%%%%%%%%%%%%%%%%%%%%%%%%%%%%%%%%%%%%%%%%%%%%%%%%%%%%%%%%%%%%





\section{Backpropagation}\label{sec:3.7}
\vspace{-0.5cm}
\noindent Soon

%%%%%%%%%%%%%%%%%%%%%%%%%%%%%%%%%%%%%%%%%%%%%%%%%%%%%%%%%%%%%%%%%%%%
\vspace{-0.3cm}
%%%%%%%%%%%%%%%%%%%%%%%%%%%%%%%%%%%%%%%%%%%%%%%%%%%%%%%%%%%%%%%%%%%%




\section{Neural Networks}\label{sec:3.8}
\vspace{-0.5cm}
\noindent Soon

%%%%%%%%%%%%%%%%%%%%%%%%%%%%%%%%%%%%%%%%%%%%%%%%%%%%%%%%%%%%%%%%%%%%
\vspace{-0.3cm}
%%%%%%%%%%%%%%%%%%%%%%%%%%%%%%%%%%%%%%%%%%%%%%%%%%%%%%%%%%%%%%%%%%%%


\subsection{Vanilla Neural Networks}\label{sec:3.8.1}
\vspace{-0.5cm}
\noindent Soon


%%%%%%%%%%%%%%%%%%%%%%%%%%%%%%%%%%%%%%%%%%%%%%%%%%%%%%%%%%%%%%%%%%%%
\vspace{-0.3cm}
%%%%%%%%%%%%%%%%%%%%%%%%%%%%%%%%%%%%%%%%%%%%%%%%%%%%%%%%%%%%%%%%%%%%


\subsection{Convolutional Neural Networks}\label{sec:3.8.2}
\vspace{-0.5cm}
\noindent Soon


%%%%%%%%%%%%%%%%%%%%%%%%%%%%%%%%%%%%%%%%%%%%%%%%%%%%%%%%%%%%%%%%%%%%
\vspace{-0.3cm}
%%%%%%%%%%%%%%%%%%%%%%%%%%%%%%%%%%%%%%%%%%%%%%%%%%%%%%%%%%%%%%%%%%%%


\subsection{Recurrent Neural Networks}\label{sec:3.8.3}
\vspace{-0.5cm}
\noindent Soon


%%%%%%%%%%%%%%%%%%%%%%%%%%%%%%%%%%%%%%%%%%%%%%%%%%%%%%%%%%%%%%%%%%%%
\vspace{-0.3cm}
%%%%%%%%%%%%%%%%%%%%%%%%%%%%%%%%%%%%%%%%%%%%%%%%%%%%%%%%%%%%%%%%%%%%

\subsection{Capsules Neural Networks}\label{sec:3.8.4}
\vspace{-0.5cm}
\noindent Soon


%%%%%%%%%%%%%%%%%%%%%%%%%%%%%%%%%%%%%%%%%%%%%%%%%%%%%%%%%%%%%%%%%%%%
\vspace{-0.3cm}
%%%%%%%%%%%%%%%%%%%%%%%%%%%%%%%%%%%%%%%%%%%%%%%%%%%%%%%%%%%%%%%%%%%%




\section{Summary}\label{sec:3.9}
\vspace{-0.5cm}
\noindent Soon

%%%%%%%%%%%%%%%%%%%%%%%%%%%%%%%%%%%%%%%%%%%%%%%%%%%%%%%%%%%%%%%%%%%%
\vspace{-0.3cm}
%%%%%%%%%%%%%%%%%%%%%%%%%%%%%%%%%%%%%%%%%%%%%%%%%%%%%%%%%%%%%%%%%%%%





%\chapter{PROPOSED METHOD}\label{ch4}
%\section{Introduction}\label{sec:4.1}
\vspace{-0.5cm}
\noindent This chapter provides a brief review of the well known sparse adaptive filters used for echo cancellation. Firstly, we present the proportionate-based adaptive algorithms as background for estimating a sparse impulse response, we then subsequently discussed the zero attracting sparse adaptive filters used in the field due to their robustness and efficiency in performance. These filters operate based on $l_1$-norm optimization such as used in CS techniques \cite{Dohono} rather than proportionate updating based approach \cite{Duttweiler}.

%%%%%%%%%%%%%%%%%%%%%%%%%%%%%%%%%%%%%%%%%%%%%%%%%%%%%%%%%%%
\vspace{-0.3cm}
%%%%%%%%%%%%%%%%%%%%%%%%%%%%%%%%%%%%%%%%%%%%%%%%%%%%%%%%%%%




\section{Review of Pipeline}\label{sec:4.2}
\vspace{-0.5cm}
\noindent Lorem ipsum dolor sit amet, consectetur adipiscing elit. In aliquam nisl in purus mattis, eget bibendum nisi ornare. Nunc quam felis, efficitur molestie posuere gravida, aliquam ut purus. Nulla scelerisque luctus nunc nec viverra. Nunc vel hendrerit lacus. Sed mauris enim, porttitor scelerisque felis et, elementum dignissim sem. Vivamus ac velit orci. Vestibulum quis elit convallis erat placerat aliquet. Vivamus ut fringilla nunc, quis sodales elit. Donec volutpat mi at auctor laoreet. Pellentesque habitant morbi tristique senectus et netus et malesuada fames ac turpis egestas. Praesent ac vestibulum lacus. Nulla facilisi. Morbi accumsan convallis molestie. Etiam ac lacinia turpis, a commodo orci.

%%%%%%%%%%%%%%%%%%%%%%%%%%%%%%%%%%%%%%%%%%%%%%%%%%%%%%%%%%%%%%%%%%%%
\vspace{-0.3cm}
%%%%%%%%%%%%%%%%%%%%%%%%%%%%%%%%%%%%%%%%%%%%%%%%%%%%%%%%%%%%%%%%%%%%

\subsection{Object Segmentation Method}\label{sec:4.2.1}
\vspace{-0.5cm}
\noindent  Lorem ipsum dolor sit amet, consectetur adipiscing elit. In aliquam nisl in purus mattis, eget bibendum nisi ornare. Nunc quam felis, efficitur molestie posuere gravida, aliquam ut purus. Nulla scelerisque luctus nunc nec viverra. Nunc vel hendrerit lacus. Sed mauris enim, porttitor scelerisque felis et, elementum dignissim sem. Vivamus ac velit orci. Vestibulum quis elit convallis erat placerat aliquet. Vivamus ut fringilla nunc, quis sodales elit. Donec volutpat mi at auctor laoreet. Pellentesque habitant morbi tristique senectus et netus et malesuada fames ac turpis egestas. Praesent ac vestibulum lacus. Nulla facilisi. Morbi accumsan convallis molestie. Etiam ac lacinia turpis, a commodo orci.


%%%%%%%%%%%%%%%%%%%%%%%%%%%%%%%%%%%%%%%%%%%%%%%%%%%%%%%%%%%%%%%%%%%%
\vspace{-0.3cm}
%%%%%%%%%%%%%%%%%%%%%%%%%%%%%%%%%%%%%%%%%%%%%%%%%%%%%%%%%%%%%%%%%%%%


\subsection{Logo Recognition Training Framework}\label{sec:4.2.2}
\vspace{-0.5cm}
\noindent  Lorem ipsum dolor sit amet, consectetur adipiscing elit. In aliquam nisl in purus mattis, eget bibendum nisi ornare. Nunc quam felis, efficitur molestie posuere gravida, aliquam ut purus. Nulla scelerisque luctus nunc nec viverra. Nunc vel hendrerit lacus. Sed mauris enim, porttitor scelerisque felis et, elementum dignissim sem. Vivamus ac velit orci. Vestibulum quis elit convallis erat placerat aliquet. Vivamus ut fringilla nunc, quis sodales elit. Donec volutpat mi at auctor laoreet. Pellentesque habitant morbi tristique senectus et netus et malesuada fames ac turpis egestas. Praesent ac vestibulum lacus. Nulla facilisi. Morbi accumsan convallis molestie. Etiam ac lacinia turpis, a commodo orci.


%%%%%%%%%%%%%%%%%%%%%%%%%%%%%%%%%%%%%%%%%%%%%%%%%%%%%%%%%%%%%%%%%%%%
\vspace{-0.3cm}
%%%%%%%%%%%%%%%%%%%%%%%%%%%%%%%%%%%%%%%%%%%%%%%%%%%%%%%%%%%%%%%%%%%%


\subsection{Logo Recognition Testing Framework}\label{sec:4.2.3}
\vspace{-0.5cm}
\noindent  Lorem ipsum dolor sit amet, consectetur adipiscing elit. In aliquam nisl in purus mattis, eget bibendum nisi ornare. Nunc quam felis, efficitur molestie posuere gravida, aliquam ut purus. Nulla scelerisque luctus nunc nec viverra. Nunc vel hendrerit lacus. Sed mauris enim, porttitor scelerisque felis et, elementum dignissim sem. Vivamus ac velit orci. Vestibulum quis elit convallis erat placerat aliquet. Vivamus ut fringilla nunc, quis sodales elit. Donec volutpat mi at auctor laoreet. Pellentesque habitant morbi tristique senectus et netus et malesuada fames ac turpis egestas. Praesent ac vestibulum lacus. Nulla facilisi. Morbi accumsan convallis molestie. Etiam ac lacinia turpis, a commodo orci.


%%%%%%%%%%%%%%%%%%%%%%%%%%%%%%%%%%%%%%%%%%%%%%%%%%%%%%%%%%%%%%%%%%%%
\vspace{-0.3cm}




\section{Review of Logos Dataset}\label{sec:4.3}
\vspace{-0.5cm}
\noindent Lorem ipsum dolor sit amet, consectetur adipiscing elit. In aliquam nisl in purus mattis, eget bibendum nisi ornare. Nunc quam felis, efficitur molestie posuere gravida, aliquam ut purus. Nulla scelerisque luctus nunc nec viverra. Nunc vel hendrerit lacus. Sed mauris enim, porttitor scelerisque felis et, elementum dignissim sem. Vivamus ac velit orci. Vestibulum quis elit convallis erat placerat aliquet. Vivamus ut fringilla nunc, quis sodales elit. Donec volutpat mi at auctor laoreet. Pellentesque habitant morbi tristique senectus et netus et malesuada fames ac turpis egestas. Praesent ac vestibulum lacus. Nulla facilisi. Morbi accumsan convallis molestie. Etiam ac lacinia turpis, a commodo orci.

%%%%%%%%%%%%%%%%%%%%%%%%%%%%%%%%%%%%%%%%%%%%%%%%%%%%%%%%%%%%%%%%%%%%
\vspace{-0.3cm}
%%%%%%%%%%%%%%%%%%%%%%%%%%%%%%%%%%%%%%%%%%%%%%%%%%%%%%%%%%%%%%%%%%%%


\section{Concept of Technologies and Frameworks}\label{sec:4.3}
\vspace{-0.5cm}
\noindent Lorem ipsum dolor sit amet, consectetur adipiscing elit. In aliquam nisl in purus mattis, eget bibendum nisi ornare. Nunc quam felis, efficitur molestie posuere gravida, aliquam ut purus. Nulla scelerisque luctus nunc nec viverra. Nunc vel hendrerit lacus. Sed mauris enim, porttitor scelerisque felis et, elementum dignissim sem. Vivamus ac velit orci. Vestibulum quis elit convallis erat placerat aliquet. Vivamus ut fringilla nunc, quis sodales elit. Donec volutpat mi at auctor laoreet. Pellentesque habitant morbi tristique senectus et netus et malesuada fames ac turpis egestas. Praesent ac vestibulum lacus. Nulla facilisi. Morbi accumsan convallis molestie. Etiam ac lacinia turpis, a commodo orci.

%%%%%%%%%%%%%%%%%%%%%%%%%%%%%%%%%%%%%%%%%%%%%%%%%%%%%%%%%%%%%%%%%%%%
\vspace{-0.3cm}
%%%%%%%%%%%%%%%%%%%%%%%%%%%%%%%%%%%%%%%%%%%%%%%%%%%%%%%%%%%%%%%%%%%%


\section{Summary}\label{sec:4.4}
\vspace{-0.5cm}
\noindent Lorem ipsum dolor sit amet, consectetur adipiscing elit. In aliquam nisl in purus mattis, eget bibendum nisi ornare. Nunc quam felis, efficitur molestie posuere gravida, aliquam ut purus. Nulla scelerisque luctus nunc nec viverra. Nunc vel hendrerit lacus. Sed mauris enim, porttitor scelerisque felis et, elementum dignissim sem. Vivamus ac velit orci. Vestibulum quis elit convallis erat placerat aliquet. Vivamus ut fringilla nunc, quis sodales elit. Donec volutpat mi at auctor laoreet. Pellentesque habitant morbi tristique senectus et netus et malesuada fames ac turpis egestas. Praesent ac vestibulum lacus. Nulla facilisi. Morbi accumsan convallis molestie. Etiam ac lacinia turpis, a commodo orci.

%%%%%%%%%%%%%%%%%%%%%%%%%%%%%%%%%%%%%%%%%%%%%%%%%%%%%%%%%%%%%%%%%%%%
\vspace{-0.3cm}
%%%%%%%%%%%%%%%%%%%%%%%%%%%%%%%%%%%%%%%%%%%%%%%%%%%%%%%%%%%%%%%%%%%%





%\chapter{SIMULATION RESULTS}\label{ch5}
%\section{Introduction}\label{sec:5.1}
\vspace{-0.5cm}
\noindent In this Chapter, we will fully show the results and experiments of our work. Let's show how the dataset will be divided and used for training CNN, as well as show what factors affect the algorithm. Comparative results with other works related to the identification and recognition of logos will be shown. Also, we thoroughly compare with the article, which was taken as the basis of the work.

%%%%%%%%%%%%%%%%%%%%%%%%%%%%%%%%%%%%%%%%%%%%%%%%%%%%%%%%%%%
\vspace{-0.3cm}
%%%%%%%%%%%%%%%%%%%%%%%%%%%%%%%%%%%%%%%%%%%%%%%%%%%%%%%%%%%

\section{Logos Dataset Preparing}\label{sec:5.2}
\vspace{-0.5cm}
\noindent Before you start training and segmenting images, you need to prepare the dataset, which will be convenient to work with, as well as to lack options where the logos are not correctly or not fully distributed between eras and packs.Also very important will be the process to remove duplicates, so when training was not overfitting. It was also important to distribute the datasets into several bundles. Another important point is that when dividing the dataset into training, validation and test sets. It is also necessary to distribute all these data correctly so that in one particular iteration of the kidneys and eras there is no oversaturation of the same class. Also, it is important to consider when training for precision and recall. After all, accuracy in some cases can play with us in a very bad joke. Class balancing can sometimes solve this problem, but counting those metrics is also very important.

\begin{table}[hbp]
	\centering
	\caption{FlickrLogos-32 content}
	\label{tab:sample}
	\begin{tabular}{cl}
		\toprule
		Definition		 					&		Images 	\\ \midrule
		Total images  	 					& 		8240   	\\
		Images containing logo instances	&		2240   	\\
		Train+Validation annotations		&		1803	\\
		Average annotations for class		&		40		\\
			(Train + Validation)			& 				\\
		Total annotation					& 		3405	\\
		\bottomrule
	\end{tabular}
\end{table}

\noindent 

%%%%%%%%%%%%%%%%%%%%%%%%%%%%%%%%%%%%%%%%%%%%%%%%%%%%%%%%%%%%%%%%%%%%
\vspace{-0.3cm}
%%%%%%%%%%%%%%%%%%%%%%%%%%%%%%%%%%%%%%%%%%%%%%%%%%%%%%%%%%%%%%%%%%%%

\section{Checking Image Segmentation and Object Proposal}\label{sec:5.3}
\vspace{-0.5cm}
\noindent In this section, we'll talk about the SS method, from the [selectivesearch] article, and how it worked for our data. Since the color gamut of all logos is very different, we tested this method on different color spaces, segment similarity measures, as well as the value that give a threshold for all images to be segmented in the future. Since it is very difficult to find the optimal size for the segmentation method, we have taken a static value that will fit the total size, in which the camera will shoot the area where the logo is supposed to be. This process is more efficient and convenient in terms of developing future applications that will use this method.

%%%%%%%%%%%%%%%%%%%%%%%%%%%%%%%%%%%%%%%%%%%%%%%%%%%%%%%%%%%%%%%%%%%%
\vspace{-0.3cm}
%%%%%%%%%%%%%%%%%%%%%%%%%%%%%%%%%%%%%%%%%%%%%%%%%%%%%%%%%%%%%%%%%%%%

\section{Experiments on Training CNN}\label{sec:5.4}
\vspace{-0.5cm}
\noindent CNN training was conducted on all parameters and various variants of the training, which were described above. This is a balancing of classes via eras and iterations on the batches, then the normalization of all batches and adding a new class, in images which do not meet the logos. We explained in detail the benefits of each of these parameters, now show the results in all parameters. Also, these parameters increase the probability of finding the logo when tested on segmented images.

%%%%%%%%%%%%%%%%%%%%%%%%%%%%%%%%%%%%%%%%%%%%%%%%%%%%%%%%%%%%%%%%%%%%
\vspace{-0.3cm}
%%%%%%%%%%%%%%%%%%%%%%%%%%%%%%%%%%%%%%%%%%%%%%%%%%%%%%%%%%%%%%%%%%%%

\section{Evaluating CNN Performance for Segmented Images}\label{sec:5.5}
\vspace{-0.5cm}
\noindent Soon

%%%%%%%%%%%%%%%%%%%%%%%%%%%%%%%%%%%%%%%%%%%%%%%%%%%%%%%%%%%%%%%%%%%%
\vspace{-0.3cm}
%%%%%%%%%%%%%%%%%%%%%%%%%%%%%%%%%%%%%%%%%%%%%%%%%%%%%%%%%%%%%%%%%%%%

\section{Application Creating}\label{sec:5.6}
\vspace{-0.5cm}
\noindent Soon

%%%%%%%%%%%%%%%%%%%%%%%%%%%%%%%%%%%%%%%%%%%%%%%%%%%%%%%%%%%%%%%%%%%%
\vspace{-0.3cm}
%%%%%%%%%%%%%%%%%%%%%%%%%%%%%%%%%%%%%%%%%%%%%%%%%%%%%%%%%%%%%%%%%%%%

\section{Summary}\label{sec:5.6}
\vspace{-0.5cm}
\noindent Soon

%%%%%%%%%%%%%%%%%%%%%%%%%%%%%%%%%%%%%%%%%%%%%%%%%%%%%%%%%%%%%%%%%%%%
\vspace{-0.3cm}
%%%%%%%%%%%%%%%%%%%%%%%%%%%%%%%%%%%%%%%%%%%%%%%%%%%%%%%%%%%%%%%%%%%%

%\chapter{CONCLUSIONS AND FUTURE WORK}\label{ch6}
%
\section{Conclusion}\label{sec:6.1}
\vspace{-0.5cm}
\noindent ISoon


\vspace{-0.5cm}
\par
\noindent Soon

\vspace{-0.5cm}
\par
\noindent Soon


\vspace{-0.5cm}
\par
\noindent Soon

%{\color[rgb]{1,0,0} This text will appear red-colored}


\newpage
\vspace{-0.3cm}
\section{Further work} \label{sec:6.2}\label{sec5.1}
\vspace{-0.5cm}
\noindent Soon




% if there is appendix
%\newpage
%\appendix \begin{centering} \large{{APPENDIX A}} \end{centering}
%\input{Appendix}

%\begin{thebibliography}{9}
%%1
\bibitem{deniz}
Deniz, P. S. R., Adaptive Filtering Algorithms and Practical Implementation,  2008, Third Ed., LLC, NY, Springer.

\bibitem{Haykins}
Haykin, S., Adaptive Filter Theory, 2002, Prentice Hall, Upper Saddle River, NJ.

%
\bibitem{Hayes}
Hayes, M. H., Statistical Digital Signal Processing and Modeling, 1996, John Wiley \& Sons. Inc., New York.

\bibitem{son}
%3
Sondhi, M. M., The History of Echo Cancellation, IEEE Signal Processing Magazine,  2006, 23, 95-102.

%4
\bibitem{Gay}
Gay, S. L., An Efficient Fast Converging Adaptive Filter for Network Echo Cancellation, Presented at the Thirty-Second Asilomar Conference on Signal, System and Amplifier, California, USA, November 1998, 394-398.

\bibitem{Gilloire}
Gilloire, A., Experiments with Sub-Band Acoustic Echo Cancellers for Teleconferencing, IEEE International Conference on  Acoustic, Speech and Signal Processing (ICASSP1987), April 1987, 2141-2144.

% 5
\bibitem{Duttweiler}
Duttweiler, D. L., Proportionate Normalised Least Mean Square Adaptation in Echo Cancelers, IEEE Transactions on Speech and Audio Processing,2000, 5, 508-518.

%
\bibitem{Romesburg}
Romesburg, E. D., Echo Canceller for Non-Linear Circuits, U.S. Patent, 5, August 1998, 796-819.


\bibitem{Benesty2}
Benesty, J., Gansler, T., Morgan, D. R., Sondhi, M. M., Gay, S. L., Advances in Network and Acoustic Echo Cancellation. Berlin, Germany: Springer-Verlag, 2001. DOI: 10.1007/978-3-662-04437-7.

\bibitem{Brookes}
Naylor, P. A., Cui, J., Brookes, M., Adaptive Algorithms for Sparse Echo Cancellation. Signal Processing, June 2006, 6, 1182-1192.



%6
 \bibitem{shukur}
 Salman, M. S., Kukrer, O., Hocanin, A., Adaptive Filtering Fundamentals and Applications, 2011, LAP LMBERT, U.S.A.

 %7
\bibitem{sayed}
Sayed, A. H., Adaptive Filters, 2008, John Wiley, Hoboken, New Jersey, 163-167.

%8
\bibitem{bellanger}
Bellanger, M. G., Adaptive Digital Filters, 2001, Second Ed., Marcel Dekker, New York.

%
\bibitem{Mathews}
Mathews, V. J., Zhenhua X., Stochastic Gradient Adaptive Filters with Gradient Adaptive Step-Sizes, International Conference on Acoustics, Speech, and Signal Processing (ICASSP1990), April 1990, 1385-1388.

\bibitem{Glover}
Widrow, B., Glover, J. R., McCool, J. M., Kaunitz, J., Williams, C. S., Hearn, R. H., Zeidler, J. R., Eugene Dong, Jr., Goodlin, R. C., Adaptive Noise Cancelling: Principles and Applications, Proceedings of the IEEE, December 1975, 1692-1716.


%15
\bibitem{Li}
 Li, Y., Gu, Y., Tang, K., Parallel NLMS Filters with Stochastic Active Taps and Step-Sizes for Sparse System Identification, IEEE International Conference on  Acoustic, Speech and Signal Processing (ICASSP2006), May 2006, 3.
 
 %
\bibitem{Chitre}
Pelekanakis, K., Chitre, M., Comparison of Sparse Adaptive Filters for Underwater Acoustic Channel Equalization/Estimation, IEEE International Conference on Communication Systems (ICCS2010), November 2010, 17, 395-399.

%35
\bibitem{gui1}
Gui, G., Peng, W., Adachi, F., Improved Adaptive Sparse Channel Estimation Based on the Least Mean Square Algorithm, IEEE Wireless Communications and Networking  Conference (WCNC), Shanghai, China, April 2013, 3105-3109.

\bibitem{Douglas}
Gay, S. L., Douglas, S. C., Normalized Natural Gradient Adaptive Filtering for Sparse and Non-Sparse Systems, IEEE International Conference on  Acoustic, Speech and Signal Processing (ICASSP2002), Orlando, Florida, March 2002, 2, 1405-1408.

\bibitem{Hero}
Chen, Y., Gu, Y., Hero, A. O., Sparse LMS for System Identification, IEEE International Conference on  Acoustic, Speech and Signal Processing (ICASSP2009), Taipei, Taiwan, April 2009, 3125-3128.

\bibitem{Qing}
Jin, J., Qing, Q., Yuantao, G., Robust Zero-Point Attraction LMS Algorithm on Near Sparse System Identification, IET Signal Processing, 2013, 3, 210-218.

\bibitem{Gu1}
Gu, Y., Jin, J., Mei, S., $l_0$-Norm Constraint LMS for Sparse System Identification, IEEE Signal Processing Letters, 1985, 9, 774-777.

\bibitem{Christina}
Christina, B., Control of a Hands-Free Telephone Set, Signal Processing, ScienceDirect, 1997, 61, 131-143.

%
\bibitem{Elko}
Elko, G. W., Diethorn, E., Gansler, T., Room Impulse Response Variation Due to Thermal Fluctuation and its Impact on Acoustic Echo Cancellation, International Workshop on Acoustic Echo Noise Control (IWAENC2003), Kyoto, Japan, September 2003, 67-70.
%
\bibitem{Peterson}
Peterson, P. M., Simulating the Response of Multiple Microphones to a Single Acoustic Source in a Reverberant Room, Journal Of the Acoustical Society of America, Nov. 1986, 5, 1527-1529.

\bibitem{Dohono}
Donoho, D. L., Compressed Sensing, IEEE Transactions on Speech and Audio Processing, 2006, 4, 1289-1306.

\bibitem{Widrow}
Widrow, B., Stearn, S. D., Adaptive Signal Processing, 1985, Printice Hall, New Jersey.





%
\bibitem{Mandic}
Mandic, D. P., A Generalized Normalized Gradient Descent Algorithm, IEEE Signal Process Letters, February 2004, 11, 115-118.

% DOI: 10.1016/S0165-1684(97)00098-4

\bibitem{Etter1}
Cheng Y. F., Etter, D. M., Analysis of an Adaptive Technique for Modeling Sparse Systems, IEEE Transaction on Acoustics, Speech, and Signal
Processing, February 1989, 2, 254-264.



\bibitem{Candes2}
Candes, E. J., Wakin, M., An Introduction To Compressive Sampling, IEEE Signal Processing Magazine, March 2008, 2, 21-30.

%13
\bibitem{Salman}
Salman, M. S., Sparse Leaky-LMS Algorithm for System Identification and its Convergence Analysis, International Journal of Adaptive Control and Signal Processing, 2008, DOI:10.1002/acs. 2428.

\bibitem{Rey}
Rey, V. L., Rey, H., Benesty, J., Tressens, S., A Family of Robust Algorithms Exploiting Sparsity in Adaptive Filters, IEEE Transactions on Audio, Speech, and Language Processing, May 2009, 4, 572-581.

%14
\bibitem{Etter}
Etter, D. M., Identification of Sparse Impulse Response System Using an Adaptive Delay Filter, IEEE International Conference on  Acoustic, Speech and Signal Processing (ICASSP1985), April 1985, 10, 1169-1172.
%DOI: 10.1109/ICASSP.1985.1168275




%16
\bibitem{Xiang}
Khong, A. W. H., Xiang, L., Doroslovacki, M., Naylor, P. A., Frequency Domain Selective Tap Adaptive Algorithm for Sparse System Identification, IEEE International Conference on  Acoustic, Speech and Signal Processing (ICASSP2008), Las Vegas, Nevada, March 2008, 229-232.
%DOI: 10.1109/ICASSP.2008.4517588

%17
\bibitem{Kawamuri}
Kawamuri, S., Hatori,  M., A Tap Selection Algorithm for Adaptive Filters, IEEE International Conference on  Acoustic, Speech and Signal Processing (ICASSP1986), April 1986, 11, 2979-2982.
% DOI: 10.1109/ICASSP.1986.1168762

%18
\bibitem{Gu2}
Su, G., Jin, J., Gu, Y., Wang, J., Performance Analysis of $l_0$-Norm Constraint Least  Mean Square Algorithm, IEEE Transactions on Signal Processing, 2011, 6, 2223-2235.

%19
\bibitem{Jin}
Gu, Y., Jin, J., Mei, S., 2010. A Stochastic Gradient Approach on Compressive Sensing Signal Reconstruction Based on Adaptive Filtering Framework, IEEE Journal of Selected Topics in Signal Processing, 2011, 2, 409-420.

%%%%%%%%%%%%%%%%%%%%%%%%%%%%%%%%%%%%%%%%%%%%%%%%%%%%%%%%%%%%%%%%%%%%%%%%%%%%%%%%
%20
\bibitem{Shi}
Shi, K., Shi, P., Convergence Analysis of Sparse LMS Algorithms with $l_1$-norm Penalty Based on White Input Signal, Signal Image and Video Processing, Springer, May 2010, 12, 3289-3293.

\bibitem{Slavakis}
Slavakis, K., Kopsinis, Y., Theodoridis, S., Adaptive Algorithm for Sparse System Identification Using Projections onto Weighted $l_1$-Balls, IEEE International Conference on Acoustics Speech and Signal Processing (ICASSP2012), March 2010, 3742-3745.


%21
\bibitem{Tong}
Wu, F. Y., Tong, F., Non-Uniform Norm Constraint LMS Algorithm for Sparse System Identification, IEEE Communications Letters 2013, 2, 385-388.

%
\bibitem{Johnson}
Martin, R. K., Sethares, W. A., Williamson, R. C., Johnson, C. R., Exploiting Sparsity in Adaptive Filters, IEEE Transactions on Signal Processing, August 2002, 8, 1883-1894.



%22
\bibitem{Harris}
Harris, R. W., Chabries  D. M., Bishop, F. A., A Variable Step (VS) Adaptive Filter Algorithm, IEEE Transactions on Acoustics, Speech and Signal Processing 1986, 2, 309-316.




%23
\bibitem{Kwong}
Kwong, R. H., Johnson, E. W., A Variable Step-Size LMS Algorithm,  IEEE Transactions on Signal Processing,1992, 7, 1633-1642.


%24
\bibitem{Brown}
Cui, J., Naylor, P. A. Brown, D. T., An Improved PNLMS Algorithm for Echo Cancellation in Packet-Switched Networks, IEEE International Conference on  Acoustic, Speech and Signal Processing (ICASSP2004), Toulouse, France, 3, May 2004, 141-144.

%25
\bibitem{Deng}
Deng, H., Doroslovacki, M., Improving Convergence of the PNLMS Algorithm for Sparse Impulse Response Identification, IEEE Signal Processing Letters,2005, 3, 181-184.

%26
\bibitem{Morgan}
Benesty, J., Morgan, D. R., Sondhi, M. M., A Better Understanding and an Improved Solution to the Specific Problems of Stereophonic Acoustic Echo
 Cancellation, IEEE International Conference on  Acoustic, Speech and Signal Processing (ICASSP1998), 2, 156-165.

%27

%28
\bibitem{Shukri}
Salman, M. S., Jahromi, N., Hocanin, A., Kukrer, O., A Zero-Attracting Variable Step-Size LMS Algorithm for Sparse System Identification, IX International Symposium on Telecommunications (BIHTEL2012), Sarajevo, Bosnia and Herzegovina, October 2012, 1-4.
%DOI: 10.1109/BIHTEL.2012.6412087

%29
\bibitem{chart}
Chartrand, R., Exact Reconstruction of Sparse Signals Via Non-Convex Minimization, IEEE Signal Processing Letters, 2007, 10, 707-710.

%30
\bibitem{rao}
Rao, B. D., Delgado, K. K., An Affine Scaling Methodology for Best Basis Selection, IEEE Transaction on Signal Processing, January 1999, 1, 187-200.


%
\bibitem{Song3}
Rao, B. D., Bongyong S., Adaptive Filtering Algorithms for Promoting Sparsity, IEEE International Conference on  Acoustic, Speech and Signal Processing (ICASSP2003), June 2003, 6, 361-364.

%31
\bibitem{mujay}
Aliyu, M. L., Alkassim, M. A., Salman, M.S., A $p$-Norm Variable Step-Size LMS Algorithm for Sparse System Identification, Signal Image and Video Processing, Springer, DOI: 10.1007/s11760-013-0610-7.

\bibitem{mujay1}
Gwadabe T. R., Aliyu M. L., Alkassim, M.A., Salman M. S., Haddad H., A New Sparse Leaky LMS Type Algorithm. IEEE 22nd Signal Processing and Communications Applications Conference (SIU 2014), Trabzon, Turkey, April 2014.

%32
\bibitem{Loganathan}
Loganathan P., Sparseness-Controlled Adaptive Algorithms for Supervised and Unsupervised System Identification, Ph.D. Thesis, 2011, Imperial College, London.



%DOI: 10.1109/WCNC.2013.6555058

%36
\bibitem{gui2}
Gui, G., Adachi, F., Improved Adaptive Sparse Channel Estimation Using Least Mean Square Algorithm, EURASIP Journal on Wireless Communications and Networking, March 2013, 1, 1-18.


\bibitem{kennedy}
Kenney, J.F., Keeping, E.S., In Mathematics of Statistics, 1962, Third Ed. Van Nostrand, Princeton.


%%%%%%%%%%%%%%%%%%%%%%%%%%%%%%%%%%%%%%%%%%%%%%%%%%%%%%%%%%%%%%%%\\\\\\\\\\\\\\\\\\\\\\\\\%%%%%%%%%%%%%%%%%%%%%%
%
\bibitem{Evans}
Evans, J. B., Xue p., Liu, B., Analysis and Implementation of Variable Step-Size Adaptive Algorithms, IEEE Transaction on Signal Processing, 8, 2517-2535.
%doi: 10.1109/78.229885

%
\bibitem{Mader}
Mader, A., Puder, H., Schmidt, G. U., Step-Size Control for Acoustic Echo Cancellation Filters-An Overview. Signal Processing, September 2000, 9, 1697-1719.



%
\bibitem{Reddy}
Reddy, V. U., Shan, T. J., Kailath, T., Application of Modified Least-Squares Algorithms to Adaptive Echo Cancellation, IEEE International Conference on  Acoustic, Speech and Signal Processing (ICASSP1983), April 1983, 8, 53-56.
%doi: 10.1109/ICASSP.1983.1172024.




%
\bibitem{Mayyas}
Mayyas, K., Aboulnasr, T., Leaky LMS Algorithm: MSE Analysis for Gaussian Data, IEEE Transactions on Signal Processing, April 1997, 4, 927-934.

%
\bibitem{Candes1}
Candes, E. J., Wakin, M., Boyd. S., Enhancing Sparsity By Reweighted $l_1$-Minimization, Journal of Fourier Analysis and Applications, October 2008, 14, 877-905.
%


%
\bibitem{Claasen}
Claasen, T., Mecklenbrauker, W., Comparison of the Convergence of Two Algorithms for Adaptive FIR Digital Filters, IEEE Transactions on Acoustics, Speech and Signal Processing, 3, June 1981, 670-678.

%
\bibitem{Hill}
Hill, S.I., Williamson, R.C., Convergence of Exponentiated Gradient Algorithms, IEEE Transactions Signal Processing, Jun 2001, 6, 1208-1215.





































%\end{thebibliography}

\end{document}

