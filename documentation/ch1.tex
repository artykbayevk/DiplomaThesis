\section{Overview} \label{sec:1.1}
\vspace{-0.5cm}
\par Object recognition and object detection are ones of the lasting and most important goals in the computer vision.Because of this problem in a wide range of applications. For example, in copyright detection, contextual advertise placement, vehicle logo for an intelligent AI-based traffic-control system and brand detecting in social media. As well as these algorithms have many applications in location recognition, advertisement, and marketing. Presently advertising is a very powerful tool for income and attracting of customers.For this reason, the analysis of the brand and mention on different resources are very important and primary tasks for business analysts.In order to captive, attractive their customers and make better decisions, companies needs for analyzing the presence of their logos in photos, videos and another type of contents. Logos help to evaluation of identity between something.[1604.06083] [1701.02620] [1711.09822]

\vspace{-0.5cm}
\par The logo mainly includes text and graphical symbols. In such cases, when the logo is in different parts of the image, the logo is inverted, the logo is distorted, as well as changed in size - recognition and definition of the logo is a very problematic and difficult task. For example, logos on their clothes, which are often deformed, which complicates its detection and recognition. [1511.02462]

\vspace{-0.5cm}
\par Recent breakthroughs in deep learning improve recognition models with very extremely minimal loss function.Models which was created for recognition based on neural networks have excellent accuracy, speed, as well as these models, have the ability to be really smart. In short, the ultimate goal of the system, which is based on the recognition model, is to create a method that defines logos accurately and continuously learn from new logos.[1711.09822]

\vspace{-0.3cm}
%%%%%%%%%%%%%%%%%%%%%%%%%%%%%%%%%%%%%%%%%%%%%%%%%%%%%%%%%%%%%%%%%%%%%%
\section{Related Work}\label{sec:1.2}
\vspace{-0.5cm}
\noindent In literature, I can find many works on logo detection, logo extraction, logo classification , logo retrieval and logo recognition.

\vspace{-0.3cm}
%%%%%%%%%%%%%%%%%%%%%%%%%%%%%%%%%%%%%%%%%%%%%%%%%%%%%%%%%%%%%%%%%%%%%%


\subsection{Logo Detection Methods}\label{sec:1.2.1}
\vspace{-0.5cm}
