
\section{Introduction}\label{sec:3.1}
\vspace{-0.5cm}
\noindent This chapter provides a review of the well-known algorithms of image segmentation, pattern recognition, exhaustive search and deep learning methods.We will also explain how the main image segmentation methods work and how they developed in computer vision. Also will demonstrated methods forward and backpropagation. Between this two process, you can see the optimization process, which try to minimize function of error. 

%%%%%%%%%%%%%%%%%%%%%%%%%%%%%%%%%%%%%%%%%%%%%%%%%%%%%%%%%%%
\vspace{-0.3cm}
%%%%%%%%%%%%%%%%%%%%%%%%%%%%%%%%%%%%%%%%%%%%%%%%%%%%%%%%%%%




\section{Computer Vision and Pattern Recognition} \label{sec:3.2}
\vspace{-0.5cm}
\noindent For a person, the perception of the outside world with your own eyes is a very simple task, be looking at any 2- or 3-dimensional object, you can safely tell about its shape and external structure.Looking at the crowd of people, the human brain can easily calculate the number of objects, can tell about their shape and condition. But what about the computer? Will the computer be able to handle the processing of objects that people see? Will the computer be able to find the difference between very similar objects? In this matter will help discipline called computer vision.This area is very closely related to signal processing, image processing, and video recording. As well as it includes machine learning with pattern recognition.Along with other Sciences like text processing and audio processing, science tries to create the ideal artificial intelligence that can think and act like a human.Image processing not only includes the transformation of images into a more comfortable and desired type but also this area along with computer vision will be able to show what is inside the image. Image processing not only includes the transformation of images into a more comfortable and desired look but also this area along with computer vision will be able to show what is inside the image. Also, this area helps in capturing movements inside the picture.[CVPR]
Understanding what exactly is happening on images and perception of this process is an important process in AI. Draw conclusions depending on what you see is a fairly simple process for a person, but not for the computer. Since a computer without any reason cannot understand the essence of the process. The problem in object recognition is the appearance of these objects in new forms or compositions because the pre-built model cannot cope with it, because it has not seen the object in this format. These new formats can be represented as an object in the expanded state, or it can be simply in motion. A huge number of new forms and aspects makes the object recognition a practically impossible task.[introTOCV]
%\vspace{-0.5cm}
%\par 

%\vspace{-0.5cm}
%\noindent 
%%%%%%%%%%%%%%%%%%%%%%%%%%%%%%%%%%%%%%%%%%%%%%%%%%%%%%%%%%%%%%%%%%%%
\vspace{-0.3cm}
%%%%%%%%%%%%%%%%%%%%%%%%%%%%%%%%%%%%%%%%%%%%%%%%%%%%%%%%%%%%%%%%%%%%


\section{Selective Search}\label{sec:3.3}
\vspace{-0.5cm}
\noindent Exhaustive search helps to find parts of the image where you want to consider the potential parts of the desired object.Although this model works well with specially selected objects, it has a number of drawbacks that significantly affect the detection of logos.After all, the search for every possible object has the ability to be impossible. To solve this, we can use selective search.To improve the whole process and the data set for testing, we can use a combined method where we will use both methods described above. Since the number of possible objects will be more and less real and possible.Diversity in this task plays an important role, as we can cover more and more possible variants of this logo in the image.Since selective search is more useful to us, it will be helpful to familiarize yourself with its dependencies.The first and most important factor is to cover as many scales as possible because the logo can be small or large. We may not warn that. After all, the situation can be quite different. Also can make problems of objects which have no clear borders, for this reason, it is necessary to look through all options of the sizes of an object.Also, it should be noted that there is no exact and general solution of searches of any objects. It is impossible to make such a general detection system. Well, at the moment of course. For this reason, you should also look at the variety of objects and their contours, which can be very important in training. Speed is an important factor when searching for possible objects in an image. After all, such systems are built to determine the objects on the camera in a short period of time.This method is exclusive in that it is possible to configure this so that it worked by concentrating on the object and not on its borders.[ssForSegmentation]

\section{Image Segmentation Methods}\label{sec:3.4}
\vspace{-0.5cm}
\noindent In practice, the importance and value is not always fully the image itself, namely what are the specific parts of the image, and sometimes just the number of channels of the image.The first and one of the most important technologies for understanding what is happening inside this image is segmentation.Since only a segmentation can be divided into important and different parts.After all, it helps to understand the image inside the image, as well as to extract useful information for us.  These aspects are extremely important for programs where image recognition is paramount.For all these reasons, it can be understood that segmentation is a very important discipline within computer vision, and in turn, segmentation has a huge number of difficulties in implementing many methods.In short, segmentation is important for recognition, because it can pull out those areas that are very important for humans. And are the basis for all methods of recognition of contours and objects.There are many types of segmentation and a huge number of places where you can use them.One of the most common methods is threshold segmentation.The basis of this algorithm creates a segmentation of the image by its regions.This method searches for a threshold by a specific criterion to create a grayscale that will be distributed from other colors. This method sets a specific threshold for pixels and depending on the condition they change from 0 to 255 in grayscale.You can also mark a method called edge segmentation. This method is particularly the fact that he refers to the saturation of gray on the borders of any object.In the discipline of computer vision and related industries, there is no single segmentation method that can work in all cases. To use the segmentation method correctly, you need to consider the advantages and disadvantages. After all, each method will lead in different ways depending on the situation and the state in the image. And it is also very important to apply the correct parameters of segmentation methods. Since the parameters play a significant role in the algorithm.[1707.02051]

\vspace{-0.5cm}
\par
\noindent Segmentation, by itself, is splitting the image into several areas, depending on their structure, size, and saturation of any particular colors. These areas can include grouped pixels, which represent the object itself, and can represent a variety of shapes, such as an arc, circle, or just a line. Developed regions can be simple lines or full-fledged objects that can have boundaries separating them from other content. Since the area of interest may not cover the entire image, we are interested in using segmentation in such cases. Segmentation has two main goals that it pursues. The first is to expand the image to the desired regions. The second task is to change the representation.Considering the simplest cases, when the interesting part of the image is very different from the rest, the segmentation will not be a problem. After all, the area of our interest, especially its color and saturation help to clearly separate it from the rest of the image. After all, the rest of the area does not have similar components as in the desired image area.But there are also severe cases where the boundaries are strongly distorted and erased as the color saturation is very similar, and the components do not differ from each other.[ch10]
%%%%%%%%%%%%%%%%%%%%%%%%%%%%%%%%%%%%%%%%%%%%%%%%%%%%%%%%%%%%%%%%%%%%
\vspace{-0.3cm}
%%%%%%%%%%%%%%%%%%%%%%%%%%%%%%%%%%%%%%%%%%%%%%%%%%%%%%%%%%%%%%%%%%%%

\subsection{Thresholding}\label{sec:3.4.1}
\vspace{-0.5cm}
\noindent  Soon

%%%%%%%%%%%%%%%%%%%%%%%%%%%%%%%%%%%%%%%%%%%%%%%%%%%%%%%%%%%%%%%%%%%%
\vspace{-0.3cm}
%%%%%%%%%%%%%%%%%%%%%%%%%%%%%%%%%%%%%%%%%%%%%%%%%%%%%%%%%%%%%%%%%%%%

\subsection{Clustering Methods}\label{sec:3.4.2}
\vspace{-0.5cm}
\noindent Soon

%%%%%%%%%%%%%%%%%%%%%%%%%%%%%%%%%%%%%%%%%%%%%%%%%%%%%%%%%%%%%%%%%%%%
\vspace{-0.3cm}
%%%%%%%%%%%%%%%%%%%%%%%%%%%%%%%%%%%%%%%%%%%%%%%%%%%%%%%%%%%%%%%%%%%%

\subsection{Compression-based methods}\label{sec:3.4.3}
\vspace{-0.5cm}
\noindent  Soon

%%%%%%%%%%%%%%%%%%%%%%%%%%%%%%%%%%%%%%%%%%%%%%%%%%%%%%%%%%%%%%%%%%%%
\vspace{-0.3cm}
%%%%%%%%%%%%%%%%%%%%%%%%%%%%%%%%%%%%%%%%%%%%%%%%%%%%%%%%%%%%%%%%%%%%

\subsection{Histogram-based methods}\label{sec:3.4.4}
\vspace{-0.5cm}
\noindent  Soon

%%%%%%%%%%%%%%%%%%%%%%%%%%%%%%%%%%%%%%%%%%%%%%%%%%%%%%%%%%%%%%%%%%%%
\vspace{-0.3cm}
%%%%%%%%%%%%%%%%%%%%%%%%%%%%%%%%%%%%%%%%%%%%%%%%%%%%%%%%%%%%%%%%%%%%

\subsection{Edge detection}\label{sec:3.4.5}
\vspace{-0.5cm}
\noindent Soon


%%%%%%%%%%%%%%%%%%%%%%%%%%%%%%%%%%%%%%%%%%%%%%%%%%%%%%%%%%%%%%%%%%%%
\vspace{-0.3cm}
%%%%%%%%%%%%%%%%%%%%%%%%%%%%%%%%%%%%%%%%%%%%%%%%%%%%%%%%%%%%%%%%%%%%

\subsection{Dual clustering method}\label{sec:3.4.6}
\vspace{-0.5cm}
\noindent Soon

%%%%%%%%%%%%%%%%%%%%%%%%%%%%%%%%%%%%%%%%%%%%%%%%%%%%%%%%%%%%%%%%%%%%
\vspace{-0.3cm}
%%%%%%%%%%%%%%%%%%%%%%%%%%%%%%%%%%%%%%%%%%%%%%%%%%%%%%%%%%%%%%%%%%%%




\section{Supervised Learning}\label{sec:3.5}
\vspace{-0.5cm}
\noindent Soon

%%%%%%%%%%%%%%%%%%%%%%%%%%%%%%%%%%%%%%%%%%%%%%%%%%%%%%%%%%%%%%%%%%%%
\vspace{-0.3cm}
%%%%%%%%%%%%%%%%%%%%%%%%%%%%%%%%%%%%%%%%%%%%%%%%%%%%%%%%%%%%%%%%%%%%




\section{Optimization}\label{sec:3.6}
\vspace{-0.5cm}
\noindent Soon

%%%%%%%%%%%%%%%%%%%%%%%%%%%%%%%%%%%%%%%%%%%%%%%%%%%%%%%%%%%%%%%%%%%%
\vspace{-0.3cm}
%%%%%%%%%%%%%%%%%%%%%%%%%%%%%%%%%%%%%%%%%%%%%%%%%%%%%%%%%%%%%%%%%%%%





\section{Backpropagation}\label{sec:3.7}
\vspace{-0.5cm}
\noindent Soon

%%%%%%%%%%%%%%%%%%%%%%%%%%%%%%%%%%%%%%%%%%%%%%%%%%%%%%%%%%%%%%%%%%%%
\vspace{-0.3cm}
%%%%%%%%%%%%%%%%%%%%%%%%%%%%%%%%%%%%%%%%%%%%%%%%%%%%%%%%%%%%%%%%%%%%




\section{Neural Networks}\label{sec:3.8}
\vspace{-0.5cm}
\noindent Soon

%%%%%%%%%%%%%%%%%%%%%%%%%%%%%%%%%%%%%%%%%%%%%%%%%%%%%%%%%%%%%%%%%%%%
\vspace{-0.3cm}
%%%%%%%%%%%%%%%%%%%%%%%%%%%%%%%%%%%%%%%%%%%%%%%%%%%%%%%%%%%%%%%%%%%%


\subsection{Vanilla Neural Networks}\label{sec:3.8.1}
\vspace{-0.5cm}
\noindent Soon


%%%%%%%%%%%%%%%%%%%%%%%%%%%%%%%%%%%%%%%%%%%%%%%%%%%%%%%%%%%%%%%%%%%%
\vspace{-0.3cm}
%%%%%%%%%%%%%%%%%%%%%%%%%%%%%%%%%%%%%%%%%%%%%%%%%%%%%%%%%%%%%%%%%%%%


\subsection{Convolutional Neural Networks}\label{sec:3.8.2}
\vspace{-0.5cm}
\noindent Soon


%%%%%%%%%%%%%%%%%%%%%%%%%%%%%%%%%%%%%%%%%%%%%%%%%%%%%%%%%%%%%%%%%%%%
\vspace{-0.3cm}
%%%%%%%%%%%%%%%%%%%%%%%%%%%%%%%%%%%%%%%%%%%%%%%%%%%%%%%%%%%%%%%%%%%%


\subsection{Recurrent Neural Networks}\label{sec:3.8.3}
\vspace{-0.5cm}
\noindent Soon


%%%%%%%%%%%%%%%%%%%%%%%%%%%%%%%%%%%%%%%%%%%%%%%%%%%%%%%%%%%%%%%%%%%%
\vspace{-0.3cm}
%%%%%%%%%%%%%%%%%%%%%%%%%%%%%%%%%%%%%%%%%%%%%%%%%%%%%%%%%%%%%%%%%%%%

\subsection{Capsules Neural Networks}\label{sec:3.8.4}
\vspace{-0.5cm}
\noindent Soon


%%%%%%%%%%%%%%%%%%%%%%%%%%%%%%%%%%%%%%%%%%%%%%%%%%%%%%%%%%%%%%%%%%%%
\vspace{-0.3cm}
%%%%%%%%%%%%%%%%%%%%%%%%%%%%%%%%%%%%%%%%%%%%%%%%%%%%%%%%%%%%%%%%%%%%




\section{Summary}\label{sec:3.9}
\vspace{-0.5cm}
\noindent Soon

%%%%%%%%%%%%%%%%%%%%%%%%%%%%%%%%%%%%%%%%%%%%%%%%%%%%%%%%%%%%%%%%%%%%
\vspace{-0.3cm}
%%%%%%%%%%%%%%%%%%%%%%%%%%%%%%%%%%%%%%%%%%%%%%%%%%%%%%%%%%%%%%%%%%%%



